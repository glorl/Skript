\cleardoublepage
\addcontentsline{toc}{section}{Symbolverzeichnis}	% F�gt das Symbolverzeichnis dem Inhaltsverzeichnis
 \markboth{Symbolverzeichnis}{Symbolverzeichnis} 	% Eintrag in Kopfzeile
\printnomenclature[12mm]	% Darstellung des Symbolverzeichnisses, 12mm Breite f�r die Variablen


\makenomenclature				% Start des Sammelns von Eintr�gen f�r das Symbolverzeichnis



\nomenclature[I]{$_{IR}$}{Test} 	  %Referenz als Index: Innenring
\nomenclature[I]{$_{AR}$}{Au�enring} 		%Referenz als Index: Aussenring
\nomenclature[I]{$_{MR}$}{Mittelring} 		%Referenz als Index: Aussenring



%===== W I C H T I G E R   H I N W E I S: =====
%
% Syntax f�r makeindex in TeXnicCenter (unter: Ausgabe -> Ausgabeprofile definieren)
% Entweder:
%		%bm.nlo -s nomencl.ist -o %bm.nls
% Oder:
%   "%bm".nlo -s nomencl.ist -o "%bm".nls
%
% %bm steht dabei f�r den Dateinamen der Hauptdatei, 





\cleardoublepage
\chapter*{Abk�rzungsverzeichnis}  
\markboth{Abk�rzungsverzeichnis}{Abk�rzungsverzeichnis} % Eintrag in Kopfzeile
\addcontentsline{toc}{section}{Abk�rzungsverzeichnis}	% F�gt den Abschnitt dem Inhaltsverzeichnis hinzu

% Abkuerzungen definieren
\label{Abkuerzungsverzeichnis}



\begin{acronym}[abcdefg]  % optionaler Parameter ist f�r die L�nge der Abk�rzunsspalte zust�ndig



\acro{AMM}{Agilent Measurement Manager}
\acro{CAD}{Computer Aided Design}
\acro{csv}{Comma Separated Values, ein Dateiformat f�r ASCII-Daten}
\acro{DMS}{Dehnungsmessstreifen}
\acro{EHD}{Elasto-Hydrodynamisch}
\acro{eps}{Encapsulated PostScript, ein Dateiformat f�r Vektorgrafiken}
\acro{FAG}{Fischers Aktien-Gesellschaft}
\acro{FEM}{Finite-Elemente-Methode}
\acro{FVA}{Forschungsvereinigung Antriebstechnik}
\acro{fig}{Figure, Dateiformat f�r Matlab-Grafiken}
\acro{GfT}{Gesellschaft f�r Tribologie}
\acro{GUI}{Graphical User Interface}
\acro{mat}{Dateiformat f�r Daten aus Matlab}
\acro{mnf}{Modal Neutral File-Format, ein Dateiformat f�r modal reduzierte K�rper}
\acro{pdf}{Portable Document Format, ein Dateiformat f�r den Dateiaustausch}
\acro{png}{Portable Network Graphics, ein Dateiformat f�r Bitmap-Grafiken}
\acro{SKF}{Svenska Kullagerfabriken (schwedisch)}


\end{acronym}

%Der Befehl \ac{betai} f�gt die Abk�rzung ein. Beim ersten Vorkommen der Abk�rzung wird zun�chst der ganze Begriff angegeben und das Akronym bzw. die Abk�rzung in Klammern.