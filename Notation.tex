\chapter*{Notation}
\markboth{Notation}{Notation} % Eintrag in Kopfzeile
\addcontentsline{toc}{chapter}{Notation}
\hspace*{-3mm}
%\begin{table}[h]
\begin{tabular}{l l}
$a$ & Skalar\\
$\underline a$ & Zeilenmatrix $ [a_1,a_2,...,a_n]^T \in \mathbb R^{n\times 1}$ \\[1mm]
$\underline A$ & Matrix $\in \mathbb R^{m\times n}$\\[1mm]
$a_i$, $A_{ij}$& Komponenten von $\underline a$ bzw. $\underline A$\\[3mm]
$\boldsymbol b$ & Tensor 1. Stufe \\[1mm]
$\boldsymbol B$ & Tensor 2. Stufe \\[1mm]
$b_{i}$, $B_{ij}$& Komponenten von $\boldsymbol b$ bzw. $\boldsymbol B$\\[5mm]
$\underline{\boldsymbol e} = [\boldsymbol e_1,\boldsymbol e_2,\boldsymbol e_3]^T$ & Basissystem \\[5mm]
$\sum_{i=1}^{n} A_{ij}\boldsymbol e_i = A_{ij}\boldsymbol e_i$& Euler'sche Summationskonvention\\[5mm]
$\underline a^T \underline a = a_i a_i$& Skalarprodukt zwischen Zeilenmatrizen\\[1mm]
$A a = \left[A_{ij}a_j\right]^T$& Matrix-Vektor-Produkt\\[5mm]
$(\underline b)^T\underline{\boldsymbol e } = b_i\boldsymbol e_i$ & Darstellung von $\boldsymbol b$ in $\underline{\boldsymbol e}$ \\[5mm]
$\boldsymbol b \cdot \boldsymbol c = b_i c_i$ & Skalarprodukt zwischen Tensoren 1. Stufe\\[1mm]
$\boldsymbol B \cdot \boldsymbol C = B_{ij} C_{ij}$ & Skalarprodukt zwischen Tensoren 2. Stufe\\[1mm]
$\boldsymbol B \boldsymbol c = B_{ij} c_{j}\boldsymbol e_i$ & Abbildung von $\boldsymbol B$ auf $\boldsymbol c$\\[1mm]
$(\boldsymbol b \otimes \boldsymbol c)\boldsymbol a = (\boldsymbol a\cdot \boldsymbol c)\boldsymbol b$ & dyadisches Produkt zwischen Tensoren 1. Stufe\\[1mm]
$\boldsymbol b \times \boldsymbol c $ & Vektorprodukt zwischen Tensoren 1. Stufe\\[5mm]
$\boldsymbol I=\boldsymbol e_i\otimes\boldsymbol e_i $ & Einheitstensor 2. Stufe\\[1mm]
$\textrm{sp}(\boldsymbol B)=\sum_{i=1}^{3}B_{ii} $ & Spur eines Tensors 2. Stufe\\[1mm]
$\textrm{det}(\boldsymbol B)$ & Determinante\\[1mm]
$\textrm{sym}(\boldsymbol B) =\frac{1}{2} \left(\boldsymbol B+\boldsymbol B^T\right)$ & Symmetrischer Anteil eines Tensors 2. Stufe\\[1mm]
$\textrm{skw}(\boldsymbol B) =\frac{1}{2} \left(\boldsymbol B-\boldsymbol B^T\right)$ & Schiefsymmetrischer Anteil eines Tensors 2. Stufe\\[5mm]
$\delta_{ij}$ & Kroneckersymbol\\[1mm]
$\boldsymbol \epsilon = \epsilon_{ijk}\boldsymbol e_i\otimes \boldsymbol e_j\otimes\boldsymbol e_k$ & Permutationstensor (3. Stufe)\\[1mm]
\end{tabular}
%\end{table}

\cleardoublepage