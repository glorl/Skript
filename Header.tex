
%===== Sprache und Struktur =============================================================
%---- Sprache und Codierung ---
%\usepackage{ngerman}								% Deutsches Sprachpaket
\usepackage[german,ngerman]{babel}					% Legt die Trennregeln fest
\usepackage[ansinew]{inputenc}			% Inputencoding
\usepackage[T1]{fontenc}				% legt fest, wie die Schriften kodiert werden
\newcommand{\changefont}[3]{				% Befehle zum einfachen �ndern der Schriftart (KOMA-Skript-Standard
\fontfamily{#1} \fontseries{#2} \fontshape{#3} \selectfont}


%---- Links ----
\usepackage[breaklinks=true]{hyperref}						% Verlinkung zwischen Zitaten und Verweisen
\usepackage[all]{hypcap}        % Sorgt dafuer, dass Links im Text auf das entsprechende Bild und nicht auf die Bildunterschrift verweisen. 
\usepackage{breakurl}						% Umbrechen zu langer URLs

\usepackage{blindtext}		% Blindtext


%===== Layout ===================================================================

%---- Seitenlayout ----
%\pagestyle{headings}

%\usepackage[paperwidth=17cm,paperheight=24cm,top=15.5mm,inner=20.5mm,outer=18.5mm,includeheadfoot]{geometry}  %Einstellen der Seitengeometrie und des Druckbereichs f�r Originalformat 17x24cm
% !!!!! Beim Originalformat Schriftgr��e 10pt verwenden
\usepackage[a4paper%,total={15.5cm,23.5cm}
,top=19mm,bottom=19mm,inner=25.5mm,outer=23mm,includeheadfoot]{geometry}	%Einstellen der Seitengeometrie und des Druckbereichs f�r A4 Format

% Inhaltsverzeichnis-Design
\usepackage{tocloft}
\renewcommand{\cftchapleader}{\cftdotfill{\cftdotsep}} 	% Punkte zwischen 
\renewcommand{\cftpartleader}{\cftdotfill{\cftdotsep}}	% Kapitelbezeichnung und 
														% Seitenzahl


\emergencystretch 1.5em                            % verhindert ueberlange Zeilen durch "Strecken" der Zeichenabstaende

\parindent0mm			% Einzug am Anfang eines neuen Abschnitts
\unitlength1mm		% L�nge 

% horizontale Ausrichtung von �berschriften 
\newlength{\secnumwidth}
\setlength{\secnumwidth}{3em}

\makeatletter
\def\@makechapterhead#1{%
	\vspace*{50\p@}%
	{\interlinepenalty\@M
		\parindent \z@ \raggedright \normalfont 
		\huge\bfseries
		\ifnum \c@secnumdepth >\m@ne
		\makebox[\secnumwidth][l]{\thechapter}%
		\fi
		#1\par\nobreak
		\vskip 40\p@
	}}
\renewcommand{\@seccntformat}[1]{\makebox[\secnumwidth][l]{\csname the#1\endcsname}}
\makeatother

%---- mathematische Symbole und Umgegungen ----
% \usepackage{tensor}                  % Tensordarstellungen
\usepackage{mathrsfs}                % F�r Sonderzeichen, z.B. Lagrange \mathscr{}
\usepackage{amsmath,amssymb,amsbsy,amsthm}  % AMS-Style


\usepackage{units}	            		 % Setzen von Einheiten nicht kursiv in Formeln
% \usepackage{undertilde}							 % Paket f�r Tilden unter Buchstaben

%---- Grafik und Farben ----
\usepackage{graphicx}						% Grafikpaket (erm�glicht unter anderem das Einbinden von .jpg-Bildern)

\usepackage[export]{adjustbox}
\usepackage{subfig}							% Paket zum Einbinden von Subfigures
%\usepackage{subfigure}
\usepackage[table]{xcolor}			% Paket zur Verwendung von farbigen Schriften, Hintergruenden etc., spezieller Zusatz Tabellenzellen unterlegen
\usepackage{pdfpages}	        % Einbinden externer PDF-Dokumente
% Zus�tzlicher Grafikpfad
\graphicspath{{Bilder/}
}
\usepackage{tikz,pgfplots}
\usepackage{wasysym}
\pgfplotsset {
    compat                   = newest,
    every tick/.append style = thin, 
    select coords between index/.style 2 args={
        x filter/.code={
            \ifnum\coordindex<#1\def\pgfmathresult{}\fi
            \ifnum\coordindex>#2\def\pgfmathresult{}\fi
        }
        }
  }
\tikzset{
  every picture/.append style={
    line join=round,
    line cap=round,
  }
}
\usetikzlibrary{positioning}
%\usepackage{KITcolors}
\usepackage{cancel}


\usepackage[framemethod=TikZ]{mdframed}

%==== Listen,Verzeichnisse und Tabellen ========================================

%---- Floats und Tabellen ----
\usepackage{float}							%	Package f"ur Floats
\usepackage[section]{placeins}	% Barriere f"ur Floats "uber sectionengrenze hinweg
\usepackage{booktabs}           % sch{\"o}nere Tabellen
\usepackage{footnpag}           % Sorgt daf�r, dass die Fussnotennummerierung mit jeder neuen Seite wieder bei 1 startet.
\usepackage{longtable}


%---- Verzeichnisse ----
\usepackage[printonlyused]%
			{acronym}					% Paket f�r die Erstellung des Abk�rzungsverzeichnisses
\usepackage{expdlist}						% Erweiterte Listenfunktionen (enumerate und Co)
\usepackage{index}              % Zur Erstellung eines Indexverzeichnisses, \makeindex
\usepackage[numbers]{natbib}    % F�r Literaturdarstellung
%\usepackage{multibib}
%\newcites{Eigene}{Eigene Publikationen}
%\newcites{BscMsc}{Betreute Abschlussarbeiten}

% Nomenklatur-Paket (Symbolvereichnis)
\usepackage[german]{nomencl}		%Nomenklatur-Paket (+ in Inhaltsverzeichnis aufgenommen)
\RequirePackage{ifthen}					% Package zum unterteilen des Symbolverzeichnisses ben�tigt
\newcommand{\nomunit}[1]{\renewcommand{\nomentryend}{\hspace*{\fill}#1}} % Erm�glicht Spalte f�r Einheit beim Formelverzeichnis
% Untergruppen im Symbolverzeichnis anlegen:
\renewcommand{\nomgroup}[1]{%
\ifthenelse{\equal{#1}{I}}{\item[\textbf{Indizes\\}]}{
\ifthenelse{\equal{#1}{S}}{\item[\textbf{Symbole}]}}}
\newcommand{\Symbol}[3]{\nomenclature[S]{$#1$}{#2 \nomunit{$\left[\unit{#3}\right]$} }} % Zum einfachen Einf�gen von Symbolen mit Makebox
\makeindex



%====== Definitions�nderungen f�r den \Autoref-Befehl ====================================

% ACHTUNG LATEX SPEZIAL: Die Eintraege muessen mit % enden sonst gibt es ueberfluessige Leerzeichen
\makeatletter  
\let\orgautoref\autoref% F�r alle, denen "`Unterunterabschnitt"' grotesk anmutet
% Definition der Singularform
\renewcommand{\autoref}%
            {\def\sectionautorefname{Abschnitt}%
            \def\subsectionautorefname{\sectionautorefname}%
            \def\subsubsectionautorefname{\sectionautorefname}%
            \def\equationautorefname{Gleichung}%
            \def\figureautorefname{Abbildung}%
            \def\subfigureautorefname{\figureautorefname}% Subfloats sind Abb.
            \def\tableautorefname{Tabelle}%
    \orgautoref}%
% Definition der Pluralform ... f�r alle, die Pluralformen verwenden wollen 
% <siehe \autoref{ADAMS}, \ref{MATLAB} und \ref{FORTRAN}> ergibt dann beispielsweise: 'siehe Abschnitte (2.3.1), (2.3.2) und (2.3.3)'
\providecommand{\autorefs}%
            {\def\sectionautorefname{Abschnitte}%
               \def\subsectionautorefname{\sectionautorefname}%
               \def\subsubsectionautorefname{\sectionautorefname}%
               \def\equationautorefname{Gleichungen}%
               \def\figureautorefname{Abbildungen}%
               \def\subfigureautorefname{\figureautorefname}%
               \def\tableautorefname{Tabellen}%
    \orgautoref}%
% Definition der im Dativ abweichenden Pluralform von Abschnitt
% <in den \autoref{ADAMS}, \ref{MATLAB} und \ref{FORTRAN}> ergibt dann beispielsweise: 'in den Abschnitten (2.3.1), (2.3.2) und (2.3.3)'
\providecommand{\autorefsd}%
            {\def\sectionautorefname{Abschnitten}%
               \def\subsectionautorefname{\sectionautorefname}%
               \def\subsubsectionautorefname{\sectionautorefname}%
               \def\equationautorefname{Gleichungen}%
                        \def\figureautorefname{Abbildungen}%
                     \def\subfigureautorefname{\figureautorefname}%
                        \def\tableautorefname{Tabellen}%
    \orgautoref}%
%---- automatische Klammern bei mathematischen Formeln ----
%%\makeatletter
\newcommand\embrace[1]{(#1)}
\renewcommand\p@equation{\expandafter\embrace}
\makeatother



%--- Ver�nderte Schriftart ------
% Latex Standard f�r Druck in Unidruckerei nicht erlaubt!

\usepackage{palatino}	% Palatino
%\usepackage{txfonts}		% Times Roman
%\usepackage{bookman}		% Bookman
%\usepackage{fouriernc} % New Century
%\usepackage{lmodern}		% Latin Modern
%\usepackage{sans}


\setkomafont{disposition}{\normalcolor\bfseries}
\usepackage{setspace}
\setstretch{1.15}

% Schriftgr��e von Bildunterschriften
\usepackage[font=footnotesize]{caption}

% Kopf- und Fu�zeile
\usepackage[]{scrlayer-scrpage}
\pagestyle{scrheadings}
\clearpairofpagestyles
% Seitenzahl (Fu�zeile)
\ofoot[\pagemark]{\pagemark}
%\ofoot[\pagemark]{Page style with \KOMAScript}
\setkomafont{pagefoot}{\normalsize}
% Kopfzeile: nicht kursiv, kleiner als Text
\ohead[]{\emph\headmark}
\setkomafont{pagehead}{\footnotesize}
%\automark[chapter]{chapter}
%\automark*[section]{}

% Fu�noten einstellen
\usepackage[hang]{footmisc}
\setlength\footnotemargin{10pt}

% Hurenkinder und Schusterjungen verhindern (f�r Druckerei)
\clubpenalty10000
\widowpenalty10000
\displaywidowpenalty=10000

\hyphenpenalty=500
\exhyphenpenalty=500

% Seiten nach unten auslaufen lassen
\raggedbottom

% Textrahmen anzeigen (Layout-check)
%\usepackage{showframe}


%%% eigene 
\usepackage{listings}    
\usepackage{subfig}
\usepackage{float}
\usepackage{tikz,pgfplots}

%\usepackage{wasysym}
\pgfplotsset {
	compat                   = newest,
	every tick/.append style = thin, 
	select coords between index/.style 2 args={
		x filter/.code={
			\ifnum\coordindex<#1\def\pgfmathresult{}\fi
			\ifnum\coordindex>#2\def\pgfmathresult{}\fi
		}
	}
}
\tikzset{
	every picture/.append style={
		line join=round,
		line cap=round,
	}
}
\usetikzlibrary{arrows,decorations.markings}
