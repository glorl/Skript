\begin{abstract}
Die Simulation der Schwingungen von Maschinen mit komplexen Geometrien erfolgt heute h�ufig in  Mehrk�rpersimulationsprogrammen wie \ADAMSO. Um in MSC/ADAMS Maschinenmodelle mit W�lzlagerungen abzubilden, wurde ein benutzerdefiniertes Element programmiert, das die Eigenschaften von W�lzlagern als generisches Maschinenelement abbildet. \\

Erster Schritt bei der Modellierung ist die Abbildung der r�umlichen Kinematik der Bauteile im W�lzlager. Aus der Durchdringung von W�lzk�rper und Laufbahn wird die r�ckstellende Kraft in Normalenrichtung unter Ber�cksichtigung des Lagerspiels ermittelt. Aus Belastung und Kontaktgeschwindigkeit lassen sich dar�ber hinaus mit Hilfe von Kennfeldern die Reibungsverluste in den einzelnen Kontaktzonen berechnen. Durch Summation �ber alle W�lzk�rper ergeben sich dann Reaktionskraft und -moment des Lagers. Zur Untersuchung von Maschinen mit gesch�digten Lagerstellen wurde dar�ber hinaus ein einfaches Erregungsmodell integriert. Durch Vergleich mit experimentellen Daten konnte das implementierte Modell validiert werden.
\end{abstract}
